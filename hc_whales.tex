\subsection{DO NOT CITE WITHOUT AUTHOR
CONSENT}\label{do-not-cite-without-author-consent}

\section{Impacts of Two Extended Foraging Events by Mammal-Eating Killer
Whales on the Population of Harbor Seals in Hood
Canal,Washington}\label{impacts-of-two-extended-foraging-events-by-mammal-eating-killer-whales-on-the-population-of-harbor-seals-in-hood-canalwashington}

\textbf{Josh M London}\\Polar Ecosystem Program\\National Marine Mammal
Laboratory, Alaska Fisheries Science Center\\National Marine Fisheries
Service, NOAA, US Dept. of Commerce\\7600 Sand Point Way NE, Seattle,
Washington, USA, 98115

\textbf{Steven J. Jeffries} (WEFW)\\\textbf{Monique Lance}
(WDFW)\\\textbf{John Durban} (SWFSC)\\\textbf{Jay M. Ver Hoef}
(NMML)\\\textbf{Paul Wade} (NMML)\\\textbf{Glenn VanBlaricom} (UW SAFS)

\subsection{Introduction}\label{introduction}

The theory that Bigg's killer whales (often referred to as
transient-type or mammal-eating killer whales) may be responsible for
declines of pinniped populations in the Eastern North Pacific has
garnered significant debate
\autocites{Springer2008}{Williams2004}{DeMaster2006}{Wade2007}. This
hypothesis stems from the conclusion that predation by a few killer
whales is responsible for a dramatic decline in abundance of sea otters
in the Aleutian Islands (Estes et al. 1998). Springer et al. (2003)
expanded the top-down effect of killer whale predation to other pinniped
species in the North Pacific. Their theories rely heavily on assumptions
regarding daily metabolic requirements of wild killer whales and their
functional response to prey populations. These assumptions and analyses
were outlined in Williams et al. (2004). Unfortunately, opportunities to
verify these assumptions with empirical data from wild populations are
limited. Bigg's killer whales are characterized by discreet behavior and
spend much of their time in remote locations not frequented by
researchers. Therefore, knowledge of killer whale intake rates is
limited to small datasets of mostly opportunistic data or extrapolations
from captive killer whales or other large terrestrial carnivores.

One might imagine an ideal situation whereby daily requirements of
killer whales could be estimated from a wild population. Under this
scenario, a group of whales would be confined to a specific geographic
area over a certain period of time. These whales would then be provided
with a known amount of prey. As long as no additional prey were added to
the area or removed by any means other than killer whale predation,
differences in prey abundance would provide an estimate for daily
energetic requirements and insights into the potential impact killer
whales might have on isolated pinniped populations throughout the
Eastern North Pacific. Additionally, observations of killer whales while
in the area would provide an independent assessment of prey consumption.

Two recent extended foraging events by killer whales in Hood Canal,
Washington are a close representation of this ideal situation and have
provided an unprecedented opportunity to empirically measure the impact
of mammal-eating killer whales on a pinniped population. Hood Canal is
an isolated 100km fjord on the west side of Puget Sound and supported an
estimated population of 1068 harbor seals in 2002. Between 2 January and
3 March, 2003, eleven mammal eating killer whales foraged exclusively
within Hood Canal. A second group of six mammal-eating whales were in
Hood Canal for 172 days in 2005.

Three separate killer whale ecotypes are present in the marine waters of
the Pacific Northwest. Fish-eating, resident-type orcas are believed to
feed exclusively on fish and predominantly on salmon. Fish-eating killer
whales have a strong matrilineal social structure and have been
extensively studied throughout Washington and British Columbia for the
past 30 years. Bigg's killer whales are known to feed exclusively on
other marine mammals (seals, sea lions, small cetaceans and some large
whales). Bigg's killer whales are less frequently observed, although an
extensive photo-identification catalog does exist and many individuals
have been photographed at least once. Offshore killer whales are a third
distinct ecotype and are most commonly found away from coastal waters.
Little is known about this ecotype, though preliminary studies indicate
they may be feeding on high trophic level fish species.

Killer whales have not had a significant presence in Hood Canal within
the past thirty years, although both Bigg's killer whales and resident
type, fish-eating killer whales have been previously observed in Hood
Canal. For both types, occurrences have been extremely rare and for less
than one or two days. A few acoustic recordings of killer whales from
U.S. Navy operations in Hood Canal have been confirmed and identified
from their unique acoustic dialect as fish-eating killer whales.

Harbor seals are reported to be one of the preferred prey items for
Bigg's killer whales, and harbor seals are the only consistently
abundant resident marine mammal species known to occur in Hood Canal.
Regular aerial and ground counts of harbor seals in Hood Canal have been
conducted since the late 1970s and the population, as a part of the
larger population of seals within the semi-enclosed marine waters of
Washington, is believed to have stabilized at near carrying capacity in
the mid-1990s (Jeffries et al. 2003). Tagging and telemetry studies
conducted in Hood Canal and other areas of Washington indicate no
significant movement of seals between areas and, therefore, any
comparison of pre and post harbor seal relative abundance is likely not
significantly compromised by emigration or immigration.

The unique nature of these two killer whale incursions to Hood Canal
provides an opportunity for empirical investigation into the predation
behavior of Bigg's killer whales and their impacts on localized pinniped
populations. In order to maximize this potential, a multi-faceted
investigation was employed. The approach can be divided into three key
areas. First, behavioral observations, mostly from the 2005 event, have
provided opportunities to directly estimate killer whale consumption and
document foraging behaviors of mammal-eating killer whales. Second, a
quantitative analysis of harbor seal aerial survey counts over time
provides a mechanism to evaluate expected population responses given the
presence of killer whales. Third, bio-energetic modeling allows a more
theoretical examination of the trophic impact of killer whales. Each
facet provides an independent evaluation of the impact of killer whale
predation on the population of harbor seals in Hood Canal. By comparing
these estimates we not only gain insight into the trophic ecology of
killer whales, but also the benefits and limitations of each approach.

\subsection{Methods}\label{methods}

\subsubsection{Behavioral Observations}\label{behavioral-observations}

All behavioral observations were conducted under the authority of
Scientific Research Permit No. 782-1719, issued by the National Marine
Fisheries Service under the authority of the Marine Mammal Protection
Act and the Endangered Species Act. Opportunities to observe killer
whales in Hood Canal during 2003 were limited to a few days. Most of the
2003 field effort focused on documenting group structure through photo
identification and understanding the spatial use of Hood Canal. All
whales were photographed from the left and right side and individual
identification was determined from identification catalogs.

Hood Canal is populated with a number of shore-side residences and,
because of its narrow, fjord-like geography, provides ample opportunity
for residents of Hood Canal to observe killer whales. Observations by a
few residents and a dedicated volunteer provided the best information on
movement and behavior of the whales in 2003.

In 2005, a coordinated effort between three research groups, in addition
to observations of local residents and volunteers, provided a better
dataset for examining killer whale foraging behavior and their spatial
use patterns. As in 2003, all whales were photographed from their left
and right side and identified through comparisons with photographic
catalogs. Nineteen boat-based observations were conducted to document
predations and movements of the killer whales within Hood Canal. Each
observation was done opportunistically given weather and researcher
availability. All observations were conducted from 19-21 foot outboard
powered vessels and all available resources were used to locate the
whales as soon as possible. Mobile phone coverage within Hood Canal
allowed local residents to quickly communicate sightings to researchers
and recent postings to internet distribution lists often provided
critical information on sightings. When recent sightings were not
available, a search transect of Hood Canal was conducted from the
research boat until the whales were located.

Once whales were located, an initial GPS location was recorded and a
trackline record was initiated. Whales were counted and visually
identified to confirm all individuals were present. In general, the
focus of the observation boat after first contact was to follow and
record confirmed predation events without altering whale behavior. Under
these circumstances, the general protocol was for the research boat to
remain approximately 100m behind the whales. Fast acceleration and `leap
frog' actions were typically avoided and all attempts were made to
minimize any effects the research boat might have on the behavior of the
whales. For some of the observation periods, other objectives, such as
collection of biopsy samples or prey remains, required temporary
departures from this protocol.

A strict protocol was employed for identification and confirmation of
predation events. Whales were closely observed for any changes in their
behavior that might indicate potential interactions with harbor seals or
other prey. All predations were confirmed by the presence of prey
remains in the water column, an oil-slick on the surface of the water or
an observation of prey remains within the mouth of a whale. Additional
behavioral clues, such as observed interactions with live seals on the
surface, and the presence of diving gulls provided further evidence of
predation activity but were not used as sole confirmation of a predation
event. The GPS location of all predation events was recorded, and each
predation event was considered complete when the whales returned to
their nominal travel behavior.

In order to extrapolate observed predations to an estimate of killer
whale consumption, observations would ideally be of equal length and
scheduled randomly across time. However, the opportunistic constraints
of our effort negated the ability to plan observations in advance.
Additionally, time to first location for any given planned observation
trip was not predictable. Therefore, all attempts were made to
approximate a random and unbiased sample of time. When possible, the
length of the observation period was pre-determined on commencement.
This was done to avoid any bias that might occur if whale behavior was
used as a determining factor. For instance, it would not be advisable to
consistently end observations after a predation event or to continue an
observation until a predation event occurred. Both situations would bias
the final estimates towards a higher consumption rate.

For each observation, a predation rate (kills/hour) was calculated from
the number of confirmed predations and the length of the observation. An
average predation rate was extrapolated across the duration of killer
whale presence in Hood Canal under two scenarios. Scenario 1 assumes
predations only occurred during daylight hours. All observations were
conducted during daylight hours only. Information on the behavior of
mammal-eating killer whales at night is limited and the only available
study suggests indications of lower activity levels at night (Baird et
al. \emph{in review}). For the daylight-only scenario, the average
predation rate was only extrapolated across hours between sunrise and
sunset. Scenario 2 was evaluated under the assumption that predation
rates observed during the day are representative of killer whale
behavior across day and night. Under this scenario, the average
predation rate was extended across all hours of the day.

\subsubsection{Generalized Linear Model}\label{generalized-linear-model}

Aerial counts done between 1996 and 2004 were assembled and incorporated
into a generalized linear model (GLM) to evaluate the impact of killer
whale consumption in 2003 on the harbor seal population of Hood Canal.
Further details on the aerial survey protocol can be found in Jeffries
et al. (2003). When available, all counts were done from photographs
taken during the aerial survey. When photographs were not taken, counts
recorded by the aerial observer were used.

Harbor seal haul out patterns are known to be influenced by tidal
height, tidal stage, time of day and day of year. Historical
observations in Hood Canal suggest harbor seals are more likely to haul
out at high tide stages in mid-afternoon and during the pupping
(August-October) and molting (September-November) seasons. All aerial
surveys in Hood Canal between 1996 and 2004 were flown between August
and November and within +/- 2 hours of high tide. Because surveys are
limited to times when seals are expected to haul out in the highest
proportions, the inclusion of tidal factors (stage and height) were not
included in the final GLM analysis.

Four hypotheses on how the Hood Canal seal population has responded to
killer whale predation can be expressed as different GLMs. The first
hypothesis suggests `no effects,' and that aerial counts are correlated
with only `day of year' and `haul-out site.' This hypothesis also
suggests the population is stable over time period from 1996 to 2002.
The second hypothesis predicts a `year effect': the population of seals
in Hood Canal is changing on an annual basis. The third hypothesis is
the `treatment' and represents a stable population between 1996 and 2000
that then changed in 2003 due to killer whale predation. The final
hypothesis is similar to the `treatment' effect but allows for growth in
the population between 2003 and 2004. To evaluate whether there was a
reduction in seal abundance in 2003, the four model variants were
compared using AIC model selection.

At the time of writing, the aerial surveys of Hood Canal for 2005 have
been completed but counts from photographs have not been finalized. Once
final counts are available, a similar GLM analysis will be conducted.

\subsubsection{Bio-Energetic Monte Carlo
Simulation}\label{bio-energetic-monte-carlo-simulation}

A bio-energetic model of killer whale consumption was developed to
estimate the predicted number of seals consumed by killer whales during
the extended stays in Hood Canal. Parameters for metabolic requirements
for killer whales were selected from published literature and
information on the caloric value of seals was derived from seals
captured in Hood Canal, caloric analysis of seals from Washington state
and values from published literature.

Caloric content of harbor seals was determined from two whole body
carcasses collected in the Grays Harbor and south Puget Sound regions of
Washington. Both animals were considered in healthy body condition at
the time of death and were provided by the Washington Department of Fish
and Wildlife. Carcasses were ground whole in the food preparation area
at United Farms in Graham, Washington. Homogenate was passed through the
grinder twice to insure complete homogenization. Four approximate four
ounce aliquots were taken from each homogenate and stored at -20 C. The
grinder was washed and cleaned between each of the carcasses to minimize
any cross contamination.

Calorimetric content was determined with a Parr 1425 semi-micro bomb
calorimeter (Parr Instrument Company, Moline, Illinois). Two 10g sub
samples from each specimen were dried to constant mass at 50 C. Constant
mass was reached when the percent change in mass was less than 0.2\% in
a 24-hour period. Sub-samples were further homogenized with mortar and
pestle and an approximate 0.10g pellet was used in the bomb calorimeter.
Caloric content was determined and converted to wet weight values based
on sample moisture loss during drying.

The equation for determining total caloric requirements of the
mammal-eating killer whales in Hood Canal is

\[latex
  \begin{aligned}
  K_{whale} \times [M_{ad.male} \times N_{m} + M_{ad.female/sub} \times N_{f} + M_{juv} \times N_{j}] \times t_{days}
  \end{aligned}
\]

where $latex K_{whale}$ is the daily caloric requirement for killer
whales in kcal/kg, $latex M_{ad.male}$ is the mass of an adult male
killer whale, $latex M_{ad.female/sub)}$ is the mass of an adult female
or sub-adult killer whale, $latex M_{juv}$ is the mass of a juvenile
killer whale, $latex N_{m}$ is the number of Adult Males, $latex N_{f}$
is the number of Adult Females/Subadults, $latex N_{j}$ is the number of
Juveniles, and $latex t_{days}$ is the number of days present in Hood
Canal.

The caloric value of harbor seals in Hood Canal can be determined from

\[latex
    \begin{aligned}
    K_{seal} \times M_{seal} \times A_{whale}
    \end{aligned}
\]

where $latex K_{seal}$ is the caloric value of a seal in kcal/kg,
$latex M_{seal}$ is the mass of a harbor seal and $latex A_{whale}$ is
the caloric assimilation value for whales.

Whale requirements divided by the caloric value of harbor seals results
in a predicted number of harbor seals consumed. This value, however,
does not accurately reflect uncertainty around any of the parameters.
Therefore, a Monte Carlo simulation was used to include uncertainty in
the final estimate.

\textbf{Table 1}: Parameter values and distributions used in the Monte
Carlo simulation of killer whale consumption of harbor seals in Hood
Canal, Washington.

\begin{longtable}[c]{@{}lllr@{}}
\hline\noalign{\medskip}
Parameter & Source & Range & Distribution
\\\noalign{\medskip}
\hline\noalign{\medskip}
Whale Requirements & Williams et al. (2004) & 54 kcal/kg/day & Normal
with s.d. = 5.5
\\\noalign{\medskip}
Adult Male Mass & & 4200-7000 kg & Uniform
\\\noalign{\medskip}
Adult Female-Subadult Mass & & 2100-3500 kg & Uniform
\\\noalign{\medskip}
Juvenile Mass & & 1365-2275 kg & Uniform
\\\noalign{\medskip}
Harbor Seal Caloric Content & Perez & 2500-3800 kcal/kg & Uniform
\\\noalign{\medskip}
Harbor Seal Mass & WDFW unpub. data & 50 kg & Normal with s.d. = 7
\\\noalign{\medskip}
Assimilation Value & Williams et al. (2004) & 0.847 & Normal with s.d. =
0.035
\\\noalign{\medskip}
Number of Days (2003) & & 59 & Fixed
\\\noalign{\medskip}
Number of Days (2005) & & 172 & Fixed
\\\noalign{\medskip}
\hline
\end{longtable}

A range of values was used for each parameter in the model (Table 1). A
value of 54 kcal/kg/day is the mean value reported by Williams et al.
(2004), Kriete (1995) and Barret-Leondard (1995) for metabolic
requirements of mammal eating killer whales. The standard deviation was
specified with a value of 5.5. Ranges for killer whale mass were
determined from reported values and consultation with other killer whale
biologists. The harbor seal caloric content range includes values
determined from the analysis of seals from Washington (this study) as
well as reported values by Perez (1990) for ringed seals. The harbor
seal mass value is a weighted average of non-pup, non-pregnant harbor
seals captured in Hood Canal between 1998 and 2002 (n=175). Non-pup,
non-pregnant weights are used to best represent the available prey
between January and June. Williams et al. (2004) reported an average
assimilation value of 0.847 and a value of 0.035 was used for the
standard deviation.

The bio-energetic model was calculated 50,000 times with new parameter
values chosen from the listed ranges each time. Consumption was
determined for each whale within each iteration to better capture
individual variability. A distribution of simulation outcomes and the
median outcome along with 2.5 and 97.5 percentiles were calculated.

\subsection{Results}\label{results}

\subsubsection{Behavioral Observations}\label{behavioral-observations-1}

The killer whales present in 2003 and 2005 represent different
individuals that are of no known relation. In 2003, the group consisted
of 11 individuals (T14, T74, T73, T73a, T73b, T73c, T77, T77a, T77b,
T123, and T123a) of which 2 were adult males (T14 and T74), 7 were
sub-adults or females and 2 were juveniles (T73c and T77b). In 2005, six
whales were present (T71, T71a, T71b, T124a, T124a1, T124a2) and the
group was composed of two adult females (T71 and T124a) and their two
offspring. With the exception of whale T14 (2003), these whales have
limited to no sighting history in Washington state. The longest and most
consistent sighting record of individuals from both groups comes from
areas of northern Southeast Alaska (pers. comm. Jan Straley, University
of Alaska Southeast, Sitka, AK).

Opportunities for detailed observations of the group in 2003 were
limited to a few boat-based observations and sighting reports from
residents of Hood Canal. All eleven whales were observed to use the
entire expanse of Hood Canal and were most often observed as either one
large group or two smaller groups of 5-6 whales. No confirmed predations
were observed during boat-based research observations, however, several
residents did report sightings of harbor seal predations and a few of
those observations were confirmed with photographic documentation.

\textbf{Figure 2.1} Box plot of observation times during 2005. The box
extents represent the time when observers were following the whales and
the dark lines represent the mid-point of the observation.

Vessel based observation effort in 2005 was significantly greater than
in 2003. Fourteen observation periods were conducted between February 2,
2005 and July 1, 2005.. The average observation period lasted 4.64 hours
with a minimum of 1.7 hours and maximum of 7.17 hours (Figure 2.1).

GPS track-lines and predation locations (Figure 2.2) clearly demonstrate
how these whales used the entire expanse of Hood Canal. Additional
locations reported by residents to Orca Network (not shown) present a
similar spatial use pattern.

\textbf{Figure 2.2} Map of North and South Hood Canal showing tracklines
from each of the boat based observations and locations of all confirmed
harbor seal predations in 2005.

A total of 18 confirmed harbor seal predations were observed during the
observation periods. One unsuccessful predation attempt on a California
sea lion was also observed, but all other predation events were
confirmed as harbor seals. It was not possible to determine the level of
individual consumption; therefore a group predation rate was calculated.
When adjusted for observation effort, the median consumption rate is
0.329 harbor seals per hour with boot-strapped 97.5 and 2.5 percentiles
of 0.465 and 0.215 harbor seals per hour, respectively. The diurnal
estimate for total consumption is 758 harbor seals consumed with a
boot-strapped confidence interval of 495-1072. The estimate of
consumption across all hours is 1358 with boot-strapped confidence
interval of 887-1921.

Behaviors observed in Hood Canal appear to be typical of other
mammal-eating killer whales. Predation events occurred over deep water,
away from any shoreline, and within a few meters of the shoreline in
relatively shallow water. The range of behaviors observed was also
variable between events. On those occasions when a predation event was
relatively short, an oil slick and small remains in the water column
were often the only indication of harbor seal presence. However,
extended predation events were often characterized by the presence of a
seal at the surface. These longer predation events often involved a
number of tail slap and ramming attacks on the seal.

\subsubsection{Generalized Linear
Model}\label{generalized-linear-model-1}

Counts from aerial surveys at five index haul-outs in Hood Canal do not
exhibit obvious signs of significant population reduction after either
of the killer whale incursions (Figure 2.3). The average count across
years from 1996 to 2000 was 684. Huber et al. (2001) have proposed a
correction factor for seals in the water of 1.56 for the inland waters
of Washington. Thus, the pre-killer whale estimate of seals in Hood
Canal is 1068.

\textbf{Figure 2.3} Box and whisker plot of aerial survey counts in Hood
Canal summed across five index haul-outs for surveys flown between
August and November from 1996 to 2004.

AIC values were determined for each GLM representing the four hypotheses
(Table 2.2). The `Treatment + Growth' model was favored with the lowest
AIC value of 1070.064. However, the AIC values for the other models
resulted in delta values of as little as 1.196 (`Year Model') and as
much as 3.233 (`Treatment Model').

\textbf{Table 2}: AIC values from four GLMs evaluating hypotheses of
harbor seal population response to killer whale predation in Hood Canal,
Washington.

\begin{longtable}[c]{@{}lr@{}}
\hline\noalign{\medskip}
Model & AIC
\\\noalign{\medskip}
\hline\noalign{\medskip}
No Effect & 1072.736
\\\noalign{\medskip}
Year Effect & 1071.260
\\\noalign{\medskip}
Treatment Only & 1073.297
\\\noalign{\medskip}
Treatment + Growth & 1070.064
\\\noalign{\medskip}
\hline
\end{longtable}

The treatment and growth coefficients calculated from the GLM under the
favored `Treatment + Growth' model suggest a treatment reduction of 24\%
(95\% CI: +3.5\% to -45.1\%) after 2003 and a growth of 49\% (95\% CI:
0\% to 123\%) in 2004. Note, for treatment effect and growth, the 95\%
confidence intervals include 0\%.

\subsubsection{Bio-energetic Monte Carlo
Simulation}\label{bio-energetic-monte-carlo-simulation-1}

Moisture content values were approximately 42 to 51 percent in the two
harbor seal carcasses processed (Table 2.3). The yearling harbor seal
carcass was recovered from southern Puget Sound and had a mass of 19 kg.
The sub-adult animal had a mass of 49 kg and was recovered from Gray's
Harbor, Washington.

The values of 2798 kcal/kg for the 49kg sub-adult and 3590 kcal/kg for
the 19kg yearling are lower values than reported for ringed seals (Perez
1990) and other pinnipeds (Williams et al. 2004).

\textbf{Table 3}: Calorimetric values determined from whole body harbor
seal carcasses recovered in Washington State.

\begin{longtable}[c]{@{}lrrr@{}}
\hline\noalign{\medskip}
Age Class & Mass (kg) & \% Moisture & kcal/kg
\\\noalign{\medskip}
\hline\noalign{\medskip}
SubAdult & 49 & 42.6 & 2798
\\\noalign{\medskip}
Yearling & 19 & 50.8 & 3590
\\\noalign{\medskip}
\hline
\end{longtable}

The bio-energetic Monte Carlo simulation for the 2003 event resulted in
a median outcome of 997 seals consumed (5th and 95th percentiles: 708,
1435). For the 2005 event, the median outcome determined from the model
was 960 (2.5 and 97.5 percentiles: 685, 1383). The distributions of
outcomes for both events are strikingly similar (Figure 2.4). The
bio-energetic model prediction compares with estimates of 758 and 1358
seals consumed for the diurnal only and all hour assumptions
respectively. The estimate from the bio-energetic model falls almost
near the midpoint of these two empirical estimates and the all-hour
consumption estimate of 1358 is within the 95\% confidence range. The
daylight only estimate falls just outside the 2.5 percentile.

Figure 1: Frequency distribution of model outputs from the bio-energetic
Monte Carlo simulation for the 2003 and 2005 killer whale incursions

\subsection{Discussion}\label{discussion}

\subsubsection{Behavioral Observations}\label{behavioral-observations-2}

The two extended foraging events by Bigg's killer whales in Hood Canal
differ in many respects from the expected behavior. Bigg's killer whales
are thought to travel in small groups and spend only a few days in one
particular area. With stays of 59 and 172 days, the Hood Canal events
represent some of the longest reported stays by Bigg's killer whales in
one area. However, the observed behaviors while in Hood Canal were not
atypical of those seen in other observations. In both 2003 and 2005, the
killer whales appear to have used the full expanse of Hood Canal as part
of their regular movement and foraging patterns. Neither group exhibited
any abnormal behaviors that might be characteristic of a group trapped
or lost.

Predation locations were not necessarily associated with harbor seal
haul-outs. Harbor seal haul-out locations in Hood Canal are
characterized by large, shallow-water, tidal expanses. The physical
characteristics of these haul-outs would provide refuge from killer
whale predation. Predation locations may better reflect harbor seal
foraging locations. The seals would be more vulnerable during foraging
activities and some of the predation locations do overlap with confirmed
foraging locations from harbor seal movement studies in Hood Canal (WDFW
unpublished data).

The estimate of total harbor seal predation during the 2005 event relies
on two key assumptions. First, that every predation event occurring
during an observation was recorded accurately. Most of the predation
activity occurs under water and out of sight of the observer. This
limitation is most obvious during shorter predation events when one
might only get a fleeting glimpse at the prey animal. During longer
events, it was more common to see a harbor seal at the surface and the
final consumption was more easily confirmed. To alleviate uncertainty in
our predation estimates, we employed a strict protocol for
identification of predations. Given other reports of harbor seal
predations being very subtle and hard to detect, estimates presented
here are probably conservative.

The second assumption critical to the calculation of total harbor seal
predation is that the predation rate observed during our observations is
representative of the un-sampled period. With the limited observation
time and opportunistic nature of the study plan, the robustness of our
estimate is lower than would normally be desirable. However, we made a
concerted effort to minimize any activity that would contribute any
significant bias to the final outcome. It would be reasonable to expect
the predation rate of the killer whale group in 2005 to change over the
nearly six month presence in Hood Canal due to changes in prey
availability or improved knowledge of the area. With only 18
observations, we are unable to examine this possibility. Fortunately,
our observation effort is spread relatively even across this period and
our estimate of average predation rate would not be overly influenced by
any temporal changes.

All predation observations conducted were diurnal and little is known
about nocturnal behaviors of mammal-eating killer whales. Baird et al.
(in review) present limited data from time-depth recorders that suggest
mammal-eating killer whales have a lower activity level at night
compared to daylight hours. Given the uncertainty about circadian
changes in foraging rates, we have chosen to present two estimates of
total harbor seal consumption. One estimate is extrapolated across just
the diurnal period, while the other is across all hours. Activity levels
of killer whales are likely influenced by more than just light-level.
Actual predation rate likely falls within the bounds of these two
estimates.

\subsubsection{Bio-Energetic Model}\label{bio-energetic-model}

Despite differences in the 2003 and 2005 events (number of days, number
of whales, individuals present, age and sex distribution), projected
consumption of harbor seals in each year based on the bio-energetic
model is strikingly similar. A likely explanation for the model
consistency is the importance of prey density and corresponding
functional response of the killer whales. As the population of seals in
Hood Canal drops due to killer whale predation, so does prey density and
prey availability. At some threshold, the cost of finding and catching
one more harbor seal in Hood Canal is no longer energetically beneficial
and the whales leave. In Hood Canal, this threshold level appears to be
a removal between 800 and 1000 seals and seems to have remained the same
across these two events. The longer presence in Hood Canal for the 2005
group is a result of their smaller group size and the absence of large
adult males, reducing their combined daily foraging requirements.

Parameters included in the bio-energetic model are reasonable given the
current state of knowledge with respect to the ecology and biology of
Bigg's killer whales. By incorporating parameter uncertainty into the
model we can better represent our understanding of killer whale
bio-energetics and the range of population impacts that could be
expected. Validation is a key aspect of any modeling exercise, and
observations conducted during the 2005 incursion have provided an
opportunity to compare the model predictions with empirical field data.
Overall, the bio-energetic model is consistent with empirical estimates
determined from field observations. The bio-energetic model and range of
parameters used appear to be appropriate predictors of killer whale
consumption of harbor seals in Hood Canal.

\subsubsection{GLM Analysis of Harbor Seal
Counts}\label{glm-analysis-of-harbor-seal-counts}

The difference between the predicted impact of killer whales on the
harbor seal population in Hood Canal and the observed population change
is unexpectedly large. The reason for the disparity is unknown at this
time. While many of the parameter values and ranges used in the
bio-energetic model are not based on empirical data, we feel these
values are reasonable based on all current knowledge of killer whale and
large mammal bio-energetics. The aerial surveys do exhibit a large
amount of variability within and between years. It is not known how much
of the variability is due to natural variation and how much is related
to sampling variability. A significant factor in this variation may be
the influence of human disturbance on haul-out patterns of seals in Hood
Canal.

The fact such a rare behavior has happened twice in the same location
within two years suggests there may be something especially attractive
about Hood Canal. The harbor seal population in Hood Canal is relatively
naïve to killer whale predation. Hood Canal is a long and narrow fjord
with deep water areas that may provide a situational advantage to the
predator. Warmer temperatures and relative quietness of the environment
in Hood Canal may also be of importance to killer whales. Any attempt to
explain why whales have chosen Hood Canal for these extended stays is
mostly speculative at this point. It does, however, seem clear from the
bio-energetic models that prey density plays a critical role in
determining the timing of departure from Hood Canal.
